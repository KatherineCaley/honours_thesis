\chapter{Discussion}

Disequilibrium of mutagenesis is a phenomenon that is widely assumed not to exist, almost never examined, but potentially of profound significance with likely important applications. In this work, I have addressed the development of statistical methods of testing for its existence and characterising the form that it takes. I have found: it exist; is greater in magnitude following perturbations on local and global scales; and differs between samples. I unified simulation study, empirical knowns, and application to a unknown case, demonstrating that even over short time frames for relatively closely related species, more than half of the genome is not a equilibrium. The basis for those conclusions and their implications will be discussed below. 

\section{Outcomes of the unified statistical development process}

\subsection{Lessons from the simulation study}

\subsection{Does mutation disequilibrium exist?}

\subsection{What is the magnitude of mutation disequilibrium?}

\subsection{Does mutation disequilibrium differ between samples?}

\section{The role of selection in influencing mutation disequilibrium}

\section{Is the Human genome at equilibrium?}

\section{Future Directions}

\section{Conclusions}