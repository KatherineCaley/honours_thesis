\chapter{Discussion}

Disequilibrium of mutagenesis is a phenomenon that is widely assumed not to exist, seldom examined, but potentially of profound significance to understanding and examining DNA sequence evolution. In this work, I have addressed the development of statistical methods of testing for its existence and characterising the form that it takes. I have found: disequilibrium of mutagenesis exists; is greater in magnitude following perturbations on local and global scales; and differs between samples. I unified simulation study and empirical knowns, allowing the application to an unknown case, in which I demonstrated that more than half of the human genome is not at equilibrium. The evidence that mutation disequilibrium exists, and the statistical tools with which I demonstrate this, are significant scientific contributions of this thesis. The basis for those conclusions and their implications will be discussed below. 

Before launching into a full discussion of this work, I will enumerate a few underpinning assumptions. First, I assume that the given sequence data is accurate and representative. Second, I assume that both the annotation of orthology and the alignment path is correct. These are standard assumptions and understanding their impact is not addressed in this thesis. 

\section{Outcomes of the unified statistical development process}
An important achievement of this thesis is the statistical tools that were developed, for which the code will be publicly available, contributing to future developments in our understanding of mutation disequilibrium. I have demonstrated the efficiency of the methods through the development process, and critically, its coupling of simulations with empirical controls.

\subsection{Lessons from simulations under the null}
The simulation study evaluated the proposed methods on corner cases of the null hypothesis. This was a foundational step in which I characterised the methods in known environments. The simulations provided valuable outcomes of how to perform the analyses, and equally importantly, when not to. 

A major outcome of the simulation study was identifying the paradoxical behaviour of the $T_{50}$ statistic. Theoretically, $T_{50}$ is a natural measure. It uses biologically interpretable units in a way conceptually related to measures employed in other disciplines. It appears, however, that a half-life statistic does not work for this type of stochastic process. Although I interrogated several possible causes, that line of inquiry is incomplete. At this point, the root of the problem is not clear, but it does not appear to be a fruitful measurement for future research. 

One imaginable explanation for the failings of $T_{50}$ is in its implicit assumption of the relationship between substitution events and chronological time. Consider that the decay of non-stationarity is exponential and halfway is determined by change in composition. When you are far from equilibrium, in the steep decline of the exponential, many of the substitutions will take you closer to equilibrium and you may reach halfway quickly. When you are close to equilibrium, more of the substitutions may look like noise. So, perhaps in the asymptotic tail, halfway can be further away. Although just a thought experiment, this paradox demonstrates that it is not obvious how to choose a metric, and our intuition is not always correct. 

The simulation study also identified the appropriate basis for conducting the hypothesis tests. This step identified where the asymptotic assumptions do not hold, a crucial discovery for the TOE (Section \ref{TOE_results}). As conventional asymptotic approximations to the TOE distribution were shown not to hold, $p$-values were estimated via parametric bootstrap. As a result, the TOE is a robust test for the existence of disequilibrium, allowing for confidence in application to natural data. 

\subsection{Does mutation disequilibrium exist?}

This validation step considered empirical applications to determine whether disequilibrium existed in natural data where there was strong biological evidence for it.

\subsubsection{High levels of mutation disequilibrium in \textit{D. melanogaster} mirror the recent loss of methylation}
Using the TOE, I estimated that 50\% of \textit{D. simulans} genes were in mutation disequilibrium (Section \ref{TOE_drosophila}). The simulans lineage had previously been shown to have a much higher level of disequilibrium, estimated by \cite{Squartini2008QuantifyingProcess} to be over 85\%. This discrepancy is due to the mistaken assumption of the asymptotic distribution by \cite{Squartini2008QuantifyingProcess}. When I applied the test statistic of \cite{Squartini2008QuantifyingProcess} to the data simulated under the null, it was abundantly clear that it did not satisfy the asymptotic distribution (Figure \ref{fig:synthetic/chi2/all-seeds}). Consequently, in application to \textit{D. simulans}, they excessively rejected the null hypothesis, ultimately causing their estimate to be incorrect.

I found a markedly higher level of mutation disequilibrium in \textit{D. melanogaster} relative to \textit{D. simulans}, 90\% and 50\% respectively. Such a substantial difference between the levels of disequilibrium is a compelling result. Where there is strong evidence of the recent loss of $^5$mC, the vast majority of genes are in mutation disequilibrium. In a closely related species where the state of mutation has not so drastically changed, only one-half of the genes are in mutation disequilibrium.

The marked difference in levels of disequilibrium supports the hypothesised impact of losing $^5$mC. In comparison to non-methylated cytosine, $^5$mC is hypermutable, with higher rates of deamination to thymine \citep{Shen1994TheDNA, Coulondre1978MolecularColi}. If not correctly repaired, what was originally a G-C pairing turns into an A-T pairing. The loss of $^5$mC in \textit{D. melanogaster} would remove this mutational bias towards creating A-T pairings. Hypothetically, any equilibrium that had been achieved with $^5$mC present likely does not constitute an equilibrium concerning the mutagenic process without $^5$mC present. Indeed, 90\% of \textit{D. melanogaster} is not in equilibrium (Section \ref{TOE_drosophila}). 

While the substantive difference in disequilibrium between the species is clear, the conjectured cause --- losing $^5$mC --- was not directly interrogated. It is possible to test the hypothesised impact of losing $^5$mC by verifying other predictions of the mechanism. For instance, the hypermutability of $^5$mC creates a mutational bias towards creating A-T base pairings. A simple supplementary analysis would look at whether the equilibrium nucleotide distribution of \textit{D. melanogaster} is then increasing in terms of GC content. 

To directly interrogate the impact made by the presence or absence of $^5$mC would require implementing a parameterised codon model. $^5$mC primarily affects cytosine found in CpG dinucleotides, consequently depleting such dinucleotides from the genome \citep{Holliday1975DNADevelopment, Bird1980DNADNA}. Extending my methods to account for this process requires a model that combines several approaches. \cite{Huttley2004ModelingMammals} introduced an approach to account for the impact of $^5$mC on CpG dinucleotides using specific additional parameters. This could be mapped onto the General Nucleotide Codon (GNC) model \citep{Kaehler2017StandardData} to deal with the confounding effect of natural selection operating on non-synonymous changes when modelling protein coding sequence. It is important to note that the large number of free parameters would make any application exceedingly slow. Furthermore, it is not obvious how exactly to parameterise a codon model, should other dinucleotides be included or just the CpG parameter? Understanding the impact of including such non-reversible context effects in a codon model, allowing the application of my statistics, is an important avenue for future development. 

Modelling evolution as a nucleotide process is a fundamental limitation of my methods and consequently a caveat to all my results. As illustrated above, the mutation process does not always act on a single nucleotide. In fact, it appears that the mutagenesis is substantially affected by the context of neighbouring bases \citep{Zhu2020MachineMutations}. Known as \gls{Simpson's Paradox} \citep{Simpson1951TheTables}, it is possible that the increased mutation disequilibrium in \textit{D. melanogaster} relative to \textit{D. simulans}, along with all other results, are limited to the nucleotide dimension and would not appear in an analysis of another dimension (e.g., a  dinucleotide process). On the other hand, it is also possible that the result could be more pronounced. To that end, I defer to the words of George Box, ``All models are wrong, [but] some are useful''. 

\subsubsection{Translocation is a putative mechanism of creating mutation disequilibrium}

I tested for the existence of mutation disequilibrium in the recently translocated \textit{Fxy} gene in \textit{M. musculus} for which, based on the conjectured model, I expected the section of the gene located in the PAR to be further from equilibrium than the X-specific section. Applying the TOE to the first six introns, I determined that all introns analysed were in mutation disequilibrium (Section \ref{Fxy_TOE}). The disequilibrium of the entire \textit{Fxy} gene is consistent with translocation of DNA sequence being a mechanism of creating disequilibrium. As explored in the introduction, there exists variation in mechanisms of mutagenesis within a genome. If a translocation event takes a sequence to an environment with a sufficiently different mutagenic process, this will hypothetically create mutation disequilibrium in the translocated sequence. Even though the first three introns of \textit{Fxy} were translocated \textit{to} an X-specific position, \textit{from} an X-specific position, there appears to be sufficient difference in the mutagenic process to have caused disequilibrium. The possibility that the cause of disequilibrium preceded the translocation could be addressed by testing for disequilibrium in the \textit{Fxy} ortholog in \textit{M. spretus}. Notably, this result validated the need to develop methods of quantifying mutation disequilibrium, since, with this result alone, the X-specific and PAR-located regions were indistinguishable.

\subsection{What is the magnitude of mutation disequilibrium?}

Another significant outcome of this thesis is the development of the novel statistic $\delta_\nabla$, a measure of the magnitude of mutation disequilibrium (Section \ref{nabla}). The effectiveness of $\delta_\nabla$ was shown by its capacity to detect predicted elevation of mutation disequilibrium following perturbations. 

\subsubsection{The magnitude of mutation disequilibrium is higher following local changes in the mutagenic environment}

Applying $\delta_\nabla$ to \textit{Fxy} separated the X-specific and PAR-located regions (Section \ref{Fxy_TOE}). The region of the gene which is located in the PAR exhibited a substantially higher magnitude of disequilibrium than the X-specific region. Given the higher levels of meiotic recombination in the PAR, recombination is a likely cause for the observed higher disequilibrium. 

Recombination may impact the evolutionary process in two main ways. Firstly, recombination may be mutagenic. A DNA double-strand break (DSB) initiates recombination \citep{Keeney2001MechanismInitiation}. The DNA around the breakpoint is then degraded and resynthesised. Resynthesis is catalysed by low fidelity DNA polymerases, possibly increasing the mutation rate \citep{Rattray2003Error-proneAhead}. Secondly, recombination is frequently associated with \acrshort{BGC}, which in mammals is biased towards fixing GC alleles \citep{Birdsell2002IntegratingEvolution, Eyre-Walker1999EvidenceDNA., Mancera2008High-resolutionYeast}. There is substantial evidence that recombination and its association with BGC is a driving force of GC rich regions of mammalian genomes \citep{Meunier2004RecombinationGenome, Berglund2009HotspotsGenes,Galtier2009GC-biasedPrimates}. The analysis of \textit{Fxy} by \cite{Montoya-Burgos2003RecombinationGenomes} found that the region within the PAR had a faster rate of evolution and higher GC content than the X-specific region, a result that is consistent with the hypothesised mutagenic impact of recombination. 

This result may be further influenced by another distinct feature of the PAR, homology with the Y chromosome. The X-linked PAR recombines with the Y-linked PAR, which is further subjected to the higher germ-line mutation rate of males \citep{Huttley2000HowMutagenesis}. Substitutions in the Y-linked \textit{Fxy} which form a mismatch with the X-linked sequence would be affected by BGC, further increasing the rate of evolution and GC content of the gene. 

I therefore conclude that the elevated magnitude of disequilibrium in the portion of \textit{Fxy} in the PAR is a result consistent with the mutagenic properties of both recombination and male-biased evolution, and ultimately their association with BGC. 

\subsubsection{The magnitude of mutation disequilibrium is higher following global changes to mutation}

To measure the magnitude of mutation disequilibrium on a global scale I applied $\delta_\nabla$ to \textit{D. melanogaster} and \textit{D. simulans} (Section \ref{TOE_drosophila}). $\delta_\nabla$ was higher in \textit{D. melanogaster} relative to \textit{D. simulans} in 82\% of genes, and significantly higher on average for all chromosomes. As the orthologs putatively have the same function, this difference likely derives from differences in the mutagenic environment. Following an event that has caused disequilibrium, as the process approaches equilibrium, $\delta_\nabla$ will approach zero. Therefore, the higher level of $\delta_\nabla$ in \textit{D. melanogaster} is consistent with disequilibrium that has been created recently. It is a reasonable hypothesis that the loss of $^5$mC is the driving force behind this elevated disequilibrium. However, an intrinsic limitation of the $\delta_\nabla$ statistic, and all methods that analyse historical processes, is the inability to determine or distinguish between causes. This possibility of other factors is not resolvable using my statistics in their current form. 

%  Consider that the methylation of DNA has a key role in the epigenetic regulation of gene expression \citep{Holliday1975DNADevelopment, Compere1981DNACells, Lieberman1983UltravioletDemethylation} such that in mammalian embryos, the loss of methylation is lethal \citep{Panning1996DNAGenes}. Presumably, \textit{D. melanogaster} has another method of regulating gene expression in place of methylation. If such a mechanism exists and has evolved recently, it may be contributing to the marked increase in mutation disequilibrium. This possibility is not resolvable using my statistics in their current form. 

% An interesting way to indicate possible causes may be through principal component analysis. 

\subsubsection{The magnitude of mutation disequilibrium appears to oscillate within an intron}

$\delta_\nabla$ appeared to oscillate with an estimated period of $\sim1,470$bp within intron 4, also observed in intron 3, although mainly at the $3'$ end. Acknowledging the possibility that this simply reflects noise in the statistic, the periodicity in intron 4 was striking. Nucleosome placement has been shown to affect the rate of substitutions in primates, observed as an oscillation of a period of about 200bp \citep{Ying2010EvidenceRepair}. This highlights the influence DNA organisation may have on evolution. 

The \textit{M. musculus} PAR has a unique chromatin structure, proposed to ensure meiotic recombination in what is a very short region of homology \citep{Kauppi2011DistinctMeiosis}. For proper segregation of the sex chromosomes, recombination must occur in the PAR during male meiosis \citep{Burgoyne1982GeneticMammals, Ellis1989TheRegion, Rappold1993TheChromosomes}, preceded by the formation of a double-strand break (DSB) \citep{Keeney2001MechanismInitiation}. In \textit{M. musculus}, the PAR appears to be structured in substantially shorter chromatin loops than in autosomes \citep{Kauppi2011DistinctMeiosis}. The chromatin loop anchor is most vulnerable to DSB \citep{Canela2017GenomeFragility}, therefore, the short loops are predicted to increase the frequency of DSB formation by increasing the number of loop anchors \citep{Acquaviva2020EnsuringRegion}. 
Indeed, the PAR has a rate of DSB 10--20 times higher than the genome average \citep{Kauppi2011DistinctMeiosis}. 

The oscillation of $\delta_\nabla$ may reflect the unique chromatin structure of the PAR. This hypothesis is based on the following observations: (1) chromatin structure is heritable \citep{Grewal2003HeterochromatinExpression, Beisel2011SilencingMechanisms}; (2) DSB occur most frequently at the anchor of a chromatin loop \citep{Canela2017GenomeFragility}; and (3) DSB influence the mutation rate \citep{Cannan2016MechanismsChromatin}. Testing this hypothesis could be achieved with a Hi-C experiment, a method used to analyse genome-wide chromatin organisation through finding genomic interactions that may be separated by a length of sequence \citep{Lieberman-Aiden2009ComprehensiveGenome}. If the peak of the $\delta_\nabla$ signal corresponds to DNA at a chromatin loop anchor, chromatin that is interacting should be equidistant from a peak of the $\delta_\nabla$ signal. 

\subsection{Does mutation disequilibrium differ between samples?}

The developed EOP tests are an important component of the set of tools as they allow for distinguishing between evolutionary processes. 

\subsubsection{The process was shown to be similar between \textit{D. melanogaster} and \textit{D. simulans} }

The process of mutation was shown to be overwhelmingly similar between orthologs of \textit{D. melanogaster} and \textit{D. simulans}, with only 5\% of genes rejecting the null of equivalence (Section \ref{TOE_drosophila}). Considering the substantial difference in disequilibrium existence and magnitude, and given one species is methylating and the other is not, this was an unexpected result. 

Assuming that the genes are evolving by the same process, is the simulans lineage closer to equilibrium because it is evolving faster? If this was the case, then we would expect for most genes that the branch length of the \textit{D. simulans} edge would be greater. However, we see the opposite, (Figure \ref{fig:drosophila_d-conv-diff}a) and for 75\% of genes, the \textit{D. melanogaster} ortholog has a longer branch length, making this explanation unlikely. 

Simpson's Paradox and power may help explain the inconsistency of equivalent processes producing strikingly different levels of disequilibrium. As explained above, the hypothesised mechanism of disequilibrium, $^5$mC, is more appropriately modelled as a dinucleotide process. Perhaps, two processes that are not discernible on the nucleotide level, may look radically different on the dinucleotide level. This raises the question of why a difference in disequilibrium was detected by the TOE and the $\delta_\nabla$ statistic? This may indicate that tEOP is a low powered test. The power of a hypothesis test is the probability of the test correctly rejecting the null hypothesis. The validation step only ensured that the test did not excessively reject a true null, but did not evaluate whether it would correctly reject an incorrect null.

I therefore conclude that the most likely explanation for these results is a combination of the low power of the test and the fact that the dominant change in the process has likely occurred on a dinucleotide level. To support this conclusion would require performing the analyses using a model that accounts for the CpG mutation process, and addressing the sensitivity of the tEOP test using positive controls. 

\subsubsection{There are significantly different evolutionary processes within the \textit{Fxy} gene}

Turning to the question of equivalence between adjacent alignments, I looked at neighbouring introns in \textit{Fxy} (Section \ref{Fxy_TOE}). As leading up to PAR each evolutionary process is different to the next, but in the PAR, the processes are the same, it is tempting to imagine that this result reflects the historical inclusion of these sequences in the PAR. The addition/attrition hypothesis of evolution of the sex chromosome proposes that genetic material is added to the PAR and is initially present on both the X and Y chromosomes, but is followed by gradual loss of the pseudoautosomal sequence on the Y chromosome  \citep{Graves1995TheGenes}. Perhaps all introns were historically included in the PAR, and thus exposed to higher levels of recombination, but the \gls{attrition} of the Y-linked PAR has slowly moved the boundary. 

The time frame of the translocation of \textit{Fxy} makes this hypothesis infeasible. The boundary of the PAR in humans is estimated to have shifted a mere 240bp due to attrition since divergence with other great apes, approximately 19\acrshort{MYA} \citep{Mensah2014PseudoautosomalPopulation}. This makes the attrition of greater than 120,000bp of the \textit{M. musculus} Y chromosome since divergence with \textit{M. spretus} 2-3MYA extremely unlikely. 

Ultimately, this result appears to be confounded by the length of the introns. The X-specific introns have a mean length of $\sim28,000$bp, an order of magnitude larger than the PAR located introns, which have a mean length of $\sim2,300$bp. The power of the test is affected by the number of substitution events that distinguish the sequences in an alignment. A longer alignment is more likely to have more substitution events, increasing the probability of rejecting an incorrect null. Therefore, it seems most likely that this result simply reflects the relative power of each test, and in turn, the length of the introns. 

\section{Is the Human genome at equilibrium?}

I now turn my attention to the evolution of the human genome. Unlike the application to empirical controls, there was no expectation of the level or form of mutation disequilibrium. 

\subsubsection{There is substantial evidence for the existence of disequilibrium in the human genome}

I estimated the proportion of human Chromosome 1 which was in disequilibrium to be $\sim50\%$ for intronic sequence and $\sim68\%$ for CDS (Section \ref{Human:TOE}). The level of disequilibrium in humans has been previously estimated to be $>99\%$ by \cite{Squartini2008QuantifyingProcess}. As explained earlier, their test statistic excessively rejects the null hypothesis as it does not satisfy the asymptotic distribution (Figure \ref{fig:synthetic/chi2/all-seeds}), ultimately overestimating the level of disequilibrium. 

It is important to acknowledge that my estimate of the proportion of human CDS in disequilibrium may be imprecise due to underlying differences between each gene. The approach of \cite{Storey2003StatisticalStudies} assumes that each data point is a result of data drawn from the same underlying distribution, in which case, you can expect that the $p$-values under the null are uniformly distributed. When you are not drawing randomly from the same distribution, that expectation is not met. In the CDS analysis, data generated under the null hypothesis was not quite uniform (Figure \ref{Human:TOE}), illustrating the impact of differences such as the length of alignment or rate of evolution. As a consequence, the properties of the estimate are not entirely understood. However, the departure from theory is small enough that I expect this estimate to be reasonably accurate. Future analysis could determine the correct proportion, achieved by randomly drawing from the true negative and true positive controls in different proportions and solving for the observed fraction. 

Acknowledging Simpson's Paradox, my results strongly suggest the existence of disequilibrium in more than half of the examined sequences. What these results do not guarantee is that the disequilibrium is entirely a consequence of changes to mutagenesis. In fact, in the case of the CDS result, I expect that the level of disequilibrium is influenced by natural selection. However, in the case of intronic sequences, I can potentially rule selection out as a major factor affecting disequilibrium. 

Determining whether selection is operating is often based on sequence variability and rates of evolution, with the underlying premise that the neutral mutation rate is uniform \citep[for examples see][]{McVicker2009WidespreadEvolution, Vitti2013DetectingData}. The challenge to that is the long-standing evidence that mutagenesis is not uniform \citep{Wolfe1989MutationGenome}. As a consequence, until we have a clearer understanding of what the true baseline of neutral processes looks like, those conclusions are in doubt \citep{Graur2013OnENCODE, Kvikstad2014StrongGenome}. 

It seems more likely that most genomic sequence, even within introns or between genes, is not functional \citep{Graur2013OnENCODE}. The demonstrated abundance of functional elements outside of protein coding sequences is low \citep{Graur2013OnENCODE} and the average length of protein binding motifs is approximately 10bp \citep{Stewart2012WhyLong}. 10bp is a minor proportion of the typical length of an intron, $\sim7,000$bp \citep{10.1093/database/baw153}. Consider conservation of sequence that encodes function as an indicator of the operation of natural selection. The rate of evolution in introns is very strong such that, for more diverged primates, introns are much less recognisably related \citep{Yi2002SlowHumans}. This indicates that the preservation of introns has been diminished in a relatively short period of time. These observations are not definitive, but they argue against the abundance of natural selection being the force behind these transitions. 

In light of the previous arguments, it seems most plausible that the level of disequilibrium in the intronic sequences is largely due to changes to mutagenesis.  This is a significant result, as it suggests that mutation disequilibrium, a phenomenon widely assumed not to exist, is a pervasive feature of genomes. Interestingly, it exists between closely related species, and without evidence of a radical change that may have caused it. 

\subsubsection{The magnitude of disequilibrium is heterogeneous along a chromosome }

The $\delta_\nabla$ statistic and its capacity to quantify disequilibrium provides an understanding of the spatial organisation of disequilibrium. The distribution of $\delta_\nabla$ along human Chromosome 1 suggests that the magnitude of disequilibrium is heterogeneous (Figure \ref{fig:primate:dconv-manhattan}). There are clear regions where the average magnitude of disequilibrium is higher than others, illustrated best by looking at the first 50M section of introns vs the next 50M section. The distribution of $\delta_\nabla$ in the autosomes of \textit{D. melanogaster} and \textit{D. simulans} does not have such contrasting sections (Figure \ref{fig:drosophila_d-conv_manhattan}). A possible explanation derives from the high level of $^5$mC in the human genome, even compared to \textit{D. simulans}. 

The spatial distribution of $\delta_\nabla$ has a striking resemblance to the frequency of CpG dinucleotides and the GC content of human Chromosome 1 (Figure \ref{fig:primate:dconv-manhattan-appendix}). A possible explanation builds upon the observation that GC content and the deamination of $^5$mC may interact and affect each other \citep{Fryxell2000CytosineIsochores, Mugal2015EvolutionaryGenomes}. The proposed model is that although the spontaneous deamination of $^5$mC reduces the GC content and depletes CpG dinucleotides,  a high GC content increases the stability of double stranded DNA, acting as a rate-limiting factor for deamination \citep{Mugal2015EvolutionaryGenomes}. This implies that the rate of deamination of $^5$mC should be lower in GC rich regions than in GC poor regions \citep{Mugal2015EvolutionaryGenomes}. This potentially creates a positive feedback loop in which low GC regions have a higher rate of deamination, depleting CpG dinucleotides and approaching equilibrium quickly, and the stability of the high GC regions slows both the rate of deamination and thus the convergence to equilibrium. This is an interesting association for which I do not have the data points to resolve. An informative next step would be to see if this association holds for other human chromosomes. If yes, is it limited to humans, or is it a feature of all species with a high level of methylation? The capacity of my methods to explore this association may further our understanding of the dynamic processes acting on genomes. 

\section{The role of purifying selection in influencing mutation disequilibrium}

Although my experimental design did not explicitly consider selection, there are two results in which its influence is visible. These are in comparisons of introns and CDS of the same gene, and comparisons between autosomes and the X chromosome. To resolve whether disequilibrium is influenced by selection as opposed to being from purely mutagenic origins requires a comparison point. This is provided in the Great Ape data by the paired data structure including introns and CDS of the same gene. I expect the introns and exons of a gene to be subject to essentially the same environment. Except, I expect selection to be operating predominantly on the CDS of a gene, making the introns a selectively neutral benchmark \citep{Graur2013OnENCODE}. 

Where the magnitude of disequilibrium is higher in the CDS than the intronic sequence may be evidence that a gene is under purifying selection. In the case where strong purifying selection is operating on a sequence, I would expect a slower rate of evolution. This will maintain disequilibrium in the sequence at a higher magnitude.  Therefore, if selection is not operating on a gene, I expect the magnitude of disequilibrium to be the same between the sequence types. Conversely, if purifying selection is operating on a gene, I expect the magnitude of disequilibrium to be higher in the CDS than in the intronic sequence.  

In the analysis of the human genome, a proportion of genes appeared to be influenced by purifying selection (Figure \ref{fig:primate:dconv-diff}). In these cases, intronic sequences appeared already closer to equilibrium. The proportion of genes for which this was the case was only a modest majority (62\%). Considering the short divergence time between the species, this is still an encouraging result, indicating that the $\delta_\nabla$ statistic is sensitive to such small differences. Analysis of alignments of humans with more diverged primates would allow a clearer understanding of the relationship between purifying selection and the magnitude of disequilibrium within a gene. 

The influence of purifying selection was also visible on a chromosomal level in the \textit{Drosophila} data. For hetero-gametic sexes, I would expect the chromosome that is hemizygous to be subjected to more stringent natural selection, because recessive deleterious alleles are exposed in the hemizygous sex. Analysis of patterns of evolution in \textit{Drosophila} indeed suggests that the efficiency of negative selection is greater on the X chromosome than the autosomes \citep{Singh2008ContrastingDrosophila}. In both \textit{D. melanogaster} and \textit{D. simulans} the mean magnitude of disequilibrium was significantly higher in the X Chromosome than the autosomes, consistent with a higher level of purifying selection acting on X-linked genes. 

\section{The Evolution of mutation}

The existence of mutation disequilibrium must be interpreted alongside the theory of evolution of mutation, as it unifies ideas of mutation, selection and drift \citep{Lynch2008TheEvolution, Lynch2010EvolutionRate}. The theory, proposed by \cite{Lynch2008TheEvolution}, states that mutation rates reflect the balance of selection operating to lower the mutation rate, opposed by random genetic drift \citep{Lynch2010EvolutionRate}. Mutators are selected against through the deleterious changes they create, and antimutators are selected for by the reduction in the deleterious mutation rate they provide. In both cases, selection must overcome the power of random genetic drift. Thus, there exists an equilibrium in each lineage that reflects this balance. In the case that an antimutator possesses a sufficient selective advantage, and thus does evolve, this would shift the equilibrium of the mutation rate. 

The loss of $^5$mC in \textit{D. melanogaster} is an example of an antimutator allele that has overcome a substantial barrier to lower the mutation rate. Critically, this recent evolution of mutation, and its shift of the equilibrium of the mutation rate, is reflected in the methods I have developed. Assuming the disequilibrium in \textit{D. melanogaster} truly echos the recent evolution of mutation, this application demonstrates that my methods can detect the perturbation of mutation rate equilibrium from the evolution of mutation. Given the availability of a vast amount of genomic data, these methods can be applied across the tree of life to help characterise how often mutators evolve. 

\section{Future Directions}

The consistency of my results with the predictions made based on biology alone demonstrates the value of my methods in understanding the disequilibrium of DNA sequence evolution, indicating that their real promise lies in future applications.

The capacity of my methods to reflect a recent antimutator evolution highlights the possibility to corroborate the underlying premises of the theory of mutation. The arguments made by \cite{Lynch2008TheEvolution, Lynch2010EvolutionRate} are supported by two key observations: the inverse relationship between the mutation rate and genome size in microbes, and the negative scaling between mutation rate and effective population size in eukaryotes. Both these observations are consistent with the random genetic drift opposing the lower bound of mutation rates, because the power of drift is ultimately dependent on the effective genetic population size \citep{Lynch2008TheEvolution, Lynch2010EvolutionRate}. 

Applying $\delta_\nabla$ to a broad range of microbes, I would expect the magnitude of mutation disequilibrium to have an inverse relationship to genome size. A similar prediction could be made about eukaryotes, except mutation disequilibrium is expected to increase with genome size. These hypotheses are based on selection having a greater capacity to select against mutators that would otherwise perturb the equilibrium when the power of drift is low. This general application will further our understanding of the interplay between selection, mutation and drift. 

The ability to detect a sequence translocated to a new environment exposes many avenues of future applications. One possible example is the detection of horizontal gene transfer (HGT) in prokaryotes. HGT is a crucial component of prokaryotic evolution, conferring functional advantages such as antibiotic resistance and enabling rapid adaption of microorganisms to a new environment \citep{Soucy2015HorizontalLife}. Current methods that rely on phylogenetic reconstruction typically lack the sensitivity to detect recent events \citep{Li2018AStrains}. Recent HGT events would likely elicit high disequilibrium, allowing for effective detection using my methods, which may uncover key components of this complex evolutionary process. 

A pertinent application would be the analysis of the SARS-CoV-2 genome. I have shown that mutation disequilibrium is a property frequently impacting protein coding sequence, in which case it will be influencing the production of new genetic variants. Applying $\delta_\nabla$ to the SARS-CoV-2 genome using a sliding window approach may reveal regions that differ in terms of their magnitude of disequilibrium. Regions with high disequilibrium indicate where the nucleotide frequency is rapidly changing. Simulating a continuation of the substitution process in these regions of high disequilibrium would suggest the type of variants (i.e., amino acid polymorphisms) likely to be produced. This will enhance our knowledge of an issue of global importance with the potential to inform future COVID-19 vaccine targets. 

The pervasive existence of mutation disequilibrium has considerable implications for research domains that use substitution models. Such applications involve uncovering the process of evolution as it manifests in biological sequences or estimating species trees. Not only do the most popular models in the field assume that the process is in equilibrium, but they have the additional constraint of reversibility, that the process is identical to how it would appear if run in reverse. Assuming stationarity and time-reversibility for data where this is not the case can weaken conclusions and even introduce bias to results \citep{Kaehler2015}. The capacity to understand the proportion of a data set where the assumption of equilibrium is untrue allows for a more informed choice of substitution model and may reduce such bias. 

\section{Conclusions}

In this thesis I have presented a set of tools that address the statistical problems of testing for disequilibrium, measuring its magnitude, and examining whether it is equivalent between samples. Through a considered experimental design I demonstrated that the tools are fit for purpose. These methods represent a significant contribution to biology, as they provide the capacity to further understand mutation disequilibrium, and in turn, how evolution is impacted by the complex and dynamic relationship between selection, mutation and drift. 

The analyses performed unveiled the status of mutation disequilibrium in biological data, showing that mutator evolution and translocation are two of many possible origins of disequilibrium. Indeed, even when considering a short time frame, more than half the human genome exhibits evidence of mutation disequilibrium. This result suggests that the assumption made by models of sequence divergence, that the process is in equilibrium, is frequently untrue. The capacity of these tools to effectively detect, measure, and distinguish mutation disequilibrium is a truly exciting prospect for further applications. In the future, these tools could be used to examine theory, detect HGT events, or predict virus variants in the pursuit of a deeper understanding of why genomes look the way they do. 