\chapter{Discussion}

Disequilibrium of mutagenesis is a phenomenon that is widely assumed not to exist, almost never examined, but potentially of profound significance in important applications. In this work, I have addressed the development of statistical methods of testing for its existence and characterising the form that it takes. I have found: it exists; is greater in magnitude following perturbations on local and global scales; and differs between samples. I unified simulation study and empirical knowns, allowing the application to an unknown case, in which I demonstrated that even over short time frames for relatively closely related species, more than half of the human genome is not a equilibrium. The basis for those conclusions and their implications will be discussed below. 

Before launching into a full discussion of this work, I will enumerate a few underpinning assumptions. I assume that the given sequence data is an accurate depiction of the true DNA sequence, and that both annotations of orthology and the alignment path are correct. These are standard assumptions. I have performed both automated (ref) and manual (ref) quality control checks on the alignments. I do not imagine this has eliminated all errors, however, understanding the impact of these assumptions is not addressed in this thesis. 

\section{Outcomes of the unified statistical development process}
An important achievement of this thesis is the statistical tools that were developed. The shown efficiency of the methods stems from the development process and its coupling of simulations with empirical controls. 

\subsection{Lessons from simulations under the null}
The simulation study evaluated the proposed methods on corner cases of the null hypothesis. This was a foundational step, in which I characterised the methods in known environments. The simulations provided valuable outcomes of how to perform the analyses, and equally importantly, when not to. 

A major outcome of the simulation study was identifying the paradoxical behaviour of the $T_{50}$ statistic. Theoretically, $T_{50}$ is a natural measure. It uses biologically interpretable units in a way conceptually related to measures employed in other disciplines. It appears, however, that a half-life statistic does not work for this type of stochastic process (Section \ref{T50_results}). Although I interrogated a number of things I thought might be the cause, that line of inquiry is incomplete. At this point, it is not entirely certain what is the problem, but it does not appear to be a fruitful measurement for future research. 

One imaginable explanation of the failings of $T_{50}$ is in its implicit assumption of the relationship between substitution events and chronological time. Consider the following thought experiment. The decay of non-stationarity is exponential. When you are far from equilibrium, many of the substitutions will take you closer to equilibrium. When you are really close to equilibrium, more of the substitutions will look like noise, and take you away from equilibrium. The halfway point is defined in terms of changes in the composition. So, perhaps in the asymptotic tail, halfway can be further away. Although just a thought experiment, this paradox demonstrates that it is not obvious how to choose a metric, and your intuition is not always correct. 

The simulation study also identified the appropriate basis for conducting the hypothesis tests. This step identified where the asymptotic assumptions do not hold, a crucial discovery for the TOE (Section \ref{TOE_results}). As conventional asymptotic approximations to the TOE distribution were shown not to hold, $p$-values were estimated via parametric bootstrap. As a result, the TOE is a robust test for the existence of mutation disequilibrium, allowing for confidence in application to natural data. 

\subsection{Does mutation disequilibrium exist?}

\subsubsection{High levels of mutation disequilibrium in \textit{D. melanogaster} mirrors recent loss of methylation}
I tackled the known cases of testing for mutation disequilibrium in \textit{D. melanogaster} and its sister taxa, \textit{D. simulans}. Using the TOE, I estimated that 50\% of \textit{D. simulans} genes were in mutation disequilibrium (Section \ref{TOE_drosophila}). The simulans lineage had previously been shown to have a much higher level of disequilibrium, estimated by \cite{Squartini2008QuantifyingProcess} to be over 85\%. 

The discrepancy with previous disequilibrium estimates is because of the mistaken assumption of the asymptotic distribution by \cite{Squartini2008QuantifyingProcess}. When I applied the test statistic of \cite{Squartini2008QuantifyingProcess} to the data simulated under the null, it was abundantly clear that it does not satisfy the asymptotic distribution (Figure \ref{fig:synthetic/chi2/all-seeds}). Consequently, in application to \textit{D. simulans}, they excessively rejected the null hypothesis, ultimately causing their estimate to be incorrect.

I found a markedly higher level of mutation disequilibrium in \textit{D. melanogaster} relative to \textit{D. simulans}, 90\% and 50\% respectively. Such a substantial difference betwen the levels of disequilibrium in the lineages is a compelling result. Where there is strong evidence of the recent loss of $^5$mC, the vast majority of genes are in mutation disequilibrium. In a closely related species where the state of mutation has not so drastically changed, only half of the genes are in mutation disequilibrium.

The marked difference in levels of disequilibrium supports the hypothesised impact of losing $^5$mC. In comparison to non-methylated cytosine, $^5$mC is hypermutable, with higher rates of deamination to thymine \citep{Shen1994TheDNA, Coulondre1978MolecularColi}. If not correctly repaired, what was originally a G-C pairing turns into a A-T pairing. The loss of $^5$mC in \textit{D. melanogaster} would remove this mutational bias towards creating A-T pairings. Hypothetically, any equilibrium that had been achieved with $^5$mC present may no longer constitute and equilibrium with respect to the mutagenic process without it. Indeed, 90\% of \textit{D. melanogaster} is not in equilibrium (Section \ref{TOE_drosophila}). 

The substantive difference in disequilibrium between the species is clear, however, the conjectured cause was not directly interrogated. It is possible to support the interpretation in other ways because the mechanism has other predictions that can be verified. The hypermutability of $^5$mC creates a mutational bias towards creating A-T base pairings. A simple supplementary analysis would look at whether the equilibrium nucleotide distribution of \textit{D. melanogaster} is then increasing in terms of GC content. 

To directly interrogate the impact made by the presence or absence of $^5$mC would require implementing a dinucleotide model. $^5$mC primarily affects cytosine found in CpG dinucleotides, consequently depleting such dinucleotides from the genome \citep{Holliday1975DNADevelopment, Bird1980DNADNA}. Extending my methods to account for this process requires the combination of a few approaches. To take into account the impact of $^5$mC requires an approach presented in \cite{Huttley2004ModelingMammals}, of adding a parameter specifically for the mutation of CpG. Additionally, as it is modelling protein coding sequence, it is critical to deal with the confounder of natural selection operating on non-synonymous changes. This is achieved by the General Nucleotide Codon (GNC) model \citep{Kaehler2017StandardData}. Mapping a CpG parameter into the GNC model would achieve the goal, however, it is important to note that the large number of free parameters would make any application exceedingly slow. Generalising my statistics for application to a dinucleotide process is an important avenue for future development. 

Modelling evolution as an independent nucleotide process is a fundamental limitation of my methods and consequently a caveat to all my results. As illustrated above, the mutation process does not act on single nucleotide. In fact, it appears that the mutagenesis is substantially affected by the context of neighbouring bases \citep{Zhu2020MachineMutations}. Known as Simpson's Paradox \citep{Simpson1951TheTables}, it is possible that the increased mutation disequilibrium in \textit{D. melanogaster} relative to \textit{D. simulans}, along with all other results, are limited to the nucleotide dimension and would not appear in an analysis in another dimension (e.g., a  dinucleotide process). On the other hand, it is also possible that the result could be more pronounced. To that end, I defer to the words of George Box, ``All models are wrong, some are useful''. 

\subsubsection{Translocation is a putative mechanism of creating mutation disequilibrium}

I tested for the existence of mutation disequilibrium in the recently translocated \textit{Fxy} gene in \textit{M. musculus}. I aimed to see if the section of the gene located in the PAR would be further from equilibrium than the X-specific section. Applying the TOE to the first six introns, I determined that all introns analysed were in mutation disequilibrium (Section \ref{Fxy_TOE}). The disequilibrium of the entire \textit{Fxy} gene suggests that translocation of DNA sequence is a mechanism of creating disequilibrium. As explored in the introduction, there exist variation in mechanisms of DNA lesions and repair systems within a genome. If a translocation event takes a sequence to an environment with a sufficiently different mutagenic process, this will hypothetically create mutation disequilibrium in the translocated sequence. Even though the first 3 introns of \textit{Fxy} were translocated to an X-specific position, from an X-specific position, there appears to be sufficient difference in the mutagenic process to have caused disequilibrium. It is possible that the disequilibrium existed prior to the translation event. This possibility could be addressed by testing for disequilibrium in the \textit{Fxy} ortholog in \textit{M. spretus}. Notably, this result validated the need to develop methods of quantifying mutation disequilibrium, for with this result alone, the X-specific and PAR-located regions were indistinguishable.

\subsection{What is the magnitude of mutation disequilibrium?}

A significant outcome of this thesis is the development of the novel statistic $\delta_\nabla$, a measure of the magnitude of mutation disequilibrium (Section \ref{nabla}). The effectiveness of $\delta_\nabla$ was shown by its capacity to detect predicted elevation of mutation disequilibrium following perturbations. I will discuss the application of $\delta_\nabla$ to positive empirical controls below.  

\subsubsection{The magnitude of mutation disequilibrium is higher following local changes in the mutagenic environment}

Applying $\delta_\nabla$ to \textit{Fxy} clearly separated the X-specific and PAR-located regions (Section \ref{Fxy_TOE}). The region of the gene which is located in the PAR exhibits a substantially higher magnitude of disequilibrium than the X-specific region. Due to the higher levels of meiotic recombination in the PAR, it is natural to consider recombination as a likely cause. 

Recombination may impact the evolution process in two main ways. Firstly, recombination may be mutagenic. A DNA double-strand break (DSB) initiates recombination \citep{Keeney2001MechanismInitiation}. The DNA around the breakpoint is then degraded and resynthesised. Resynthesis is catalysed by low fidelity DNA polymerases, possibly increasing the mutation rate \citep{Rattray2003Error-proneAhead}. Secondly, recombination is frequently associated with allelic conversion, where one allele at a mismatch between homologues is converted to the other allele. In mammals, the conversion appears to have a bias towards fixing GC alleles, referred to as biased gene conversion (BGC) \citep{Birdsell2002IntegratingEvolution, Eyre-Walker1999EvidenceDNA., Mancera2008High-resolutionYeast}. There is substantial evidence that recombination and its association with BGC is a driving force of GC rich regions of genomes \citep{Meunier2004RecombinationGenome, Berglund2009HotspotsGenes,Galtier2009GC-biasedPrimates}. The analysis of \textit{Fxy} by \cite{Montoya-Burgos2003RecombinationGenomes} found that the region within the PAR had a faster rate of evolution and higher GC content than the X-specific region, a result that is consistent with the hypothesised mutagenic impact of recombination. 

It is possible that this result is further influenced by another distinct feature of the PAR, homology with the Y chromosome. The X-linked PAR recombines with the Y-linked sequence, which is further subjected to the higher germ-line mutation rate of males \citep{Huttley2000HowMutagenesis}. Substitutions in the Y-linked \textit{Fxy} which form a mismatch with the X-linked sequence would be affected by BGC, further increasing the rate of evolution and GC content of the gene. 

I therefore conclude that the elevated magnitude of disequilibrium in the portion of \textit{Fxy} in the PAR is a result consistent with the mutagenic properties of both recombination and male-biased evolution, and ultimately their association with BGC. 

\subsubsection{The magnitude of mutation disequilibrium is higher following global changes to mutation}

To measure the magnitude of mutation disequilibrium on a global scale I applied $\delta_\nabla$ to \textit{D. melanogaster} and \textit{D. simulans} genomes (Section \ref{TOE_drosophila}). The mean of $\delta_\nabla$ was significantly higher in \textit{D. melanogaster} relative to \textit{D. simulans}, consistent across all chromosomes. In fact, $82$\% of genes had a higher magnitude of mutation disequilibrium in \textit{D. melanogaster}. Following an event that has caused disequilibrium, as the process approaches equilibrium, $\delta_\nabla$ will asymptote to zero. Therefore, the level of $\delta_\nabla$ in \textit{D. melanogaster} suggests that disequilibrium has been recently created. It is a reasonable hypothesis that the loss of $^5$mC is the driving force behind this elevated disequilibrium, however, it is difficult to rule out other factors. 

An intrinsic limitation of the $\delta_\nabla$ statistic, and all methods that analyse historical processes, is the inability to determine or distinguish between causes. Considering that the methylation of DNA has a key role in the epigenetic regulation of gene expression \citep{Holliday1975DNADevelopment, Compere1981DNACells, Lieberman1983UltravioletDemethylation}. Furthermore, in mammalian embryos, the loss of methylation is lethal \citep{Panning1996DNAGenes}. Such is pure speculation, however, \textit{D. melanogaster} presumably has another method of regulating gene expression in lieu of methylation. If such a mechanism exists and has evolved recently, it may be contributing to the marked increase in mutation disequilibrium. This possibility is not resolvable using my statistics in their current form. 

An interesting way to indicate possible causes may be through principal component analysis. pca is this. the equilibrium distribution is this 

\subsubsection{The magnitude of mutation disequilibrium is variable within an intron}

$\delta_\nabla$ appeared to oscillate with a period of $\sim1,470$bp within intron 4, this was also observed to an extent in intron 3, although mainly at the $3'$ end. Acknowledging the possibility that this simply reflects noise in the statistic, the periodicity in intron 4 was striking. Nucleosome placement has been shown to affect the rate of substitutions in primates, observed as an oscillation of a period of about 200bp \citep{Ying2010EvidenceRepair}. This highlights the influence of DNA organisation in creating regionally distinct environments and thus the potential impact it may have on evolution. 

The \textit{M. musculus} PAR has a unique chromatin structure, proposed to ensure meiotic recombination in what is a very short region of homology \citep{Kauppi2011DistinctMeiosis}. For proper segregation of the sex chromosomes, crossing over must occur in the PAR during male meiosis (Burgoyne  1982;  Ellis  andGoodfellow 1989; Rappold 1993 \citep{Perry2001AMice}). Meiotic recombination is preceded by the formation of a double-strand break (DSB) \citep{Keeney2001MechanismInitiation}. In \textit{M. musculus}, the PAR appears to be structured in substantially shorter chromatin loops than autosomes, predicted to increase the frequency of DSB formation \citep{Kauppi2011DistinctMeiosis, Acquaviva2020EnsuringRegion}. Indeed, the PAR has a rate of DSB 10-20 times higher than the genome average \citep{Kauppi2011DistinctMeiosis}. 

The oscillation of $\delta_\nabla$ may reflect the unique chromatin structure of the PAR.
This hypothesis is based on the following assumptions: (1), chromatin structure is largely preserved between generations (heritable); (2) DSB occur at a certain position with respect to chromatin structure; (3) DSB influence the mutation rate. To test this hypothesis would require showing that the period of the signal aligns with the expected chromatin loop size. This could be achieved by mapping the interaction of chromatin in the PAR, commonly performed using ChIP-sequencing experiment (ref), in which I would predict that interactions would occur equidistant from a peak of the $\delta_\nabla$ signal.

\subsection{Does mutation disequilibrium differ between samples?}

I developed two EOP tests to tackle the statistical problem of whether two related sequences were evolving by an equivalent process. The EOP tests are crucial components of the contribution of my methods to biology. 

\subsubsection{The process was shown to be similar between \textit{D. melanogaster} and \textit{D. simulans} }

The process of mutation was shown to be overwhelmingly similar between orthologs of \textit{D. melanogaster} and \textit{D. simulans}, with only 5\% of genes rejecting the null of equivalence (Section \ref{TOE_drosophila}). Considering the substantial difference in disequilibrium existence and magnitude, and the fact that one species is methylating and the other is not, this was an unexpected result. 

Assuming that the genes actually are evolving by the same process, is the simulans lineage closer to equilibrium because it is evolving faster? If this was the case, then we would expect for most genes that the branch length of the \textit{D. simulans} edge would be greater. However, we actually see the opposite, (Figure \ref{fig:drosophila_d-conv-diff}a) and for 75\% of genes, the \textit{D. melanogaster} ortholog has a longer branch length, making this explanation unlikely. 

Simpson's paradox and power may help explain the inconsistency of equivalent processes producing strikingly different levels of disequilibrium. As explained above, the hypothesised mechanism of disequilibrium, $^5$mC, is more appropriately modelled as a dinucleotide process. Perhaps, two processes that are not discernible on a nucleotide level, may be radically different on the dinucleotide level. This raises the question of how come a difference was detected by the TOE and the $\delta_\nabla$ statistic? It is also possible that the temporal EOP is a low powered test. The power of a hypothesis test is the probability of the test correcting rejecting the null hypothesis. The validation step only ensured that the test did not reject a true null, but did not evaluate whether it would correct reject the null in cases where it was not true. 

I therefore conclude that the most likely explanation for these results is a combination between the low power of the test and the fact that the dominant change in the process has occurred on a dinucleotide level. To support this conclusion would require performing the analyses using a dinucleotide process and developing positive controls to address the sensitivity of the tEOP test. 

\subsubsection{There are significantly different evolutionary processes within the \textit{Fxy} gene}

Turning to the question of equivalence between adjacent alignment, I looked at neighbouring introns in \textit{Fxy}. I found that the introns within the PAR were evolving by the same process, yet all other adjacent comparisons were evolving by different processes (Section \ref{Fxy_TOE}). 

It is tempting to imagine that this result reflects the historical inclusion of these sequences in the PAR. The addition/attrition hypothesis of evolution of the sex chromosome proposes that genetic material is added to the PAR and is initially present on both the X and Y chromosomes, but is followed by attrition of the pseudoautosomal sequence on the Y chromosome  \citep{Graves1995TheGenes}. Leading up to PAR each process is different to the next, but in the PAR, the processes are the same. Perhaps we are seeing the influence of the amount of time each intron spent in the PAR. 

The time frame of the translocation of \textit{Fxy} makes this hypothesis infeasible. The boundary of the PAR in humans is estimated to have shifted a mere 240bp due to attrition since divergence with other great apes, approximately 19MYA \citep{Mensah2014PseudoautosomalPopulation}. This makes the attrition of greater than 120,000bp of the \textit{M. musculus} Y chromosome in the 2-3 million years (MY) since divergence with \textit{M. spretus} extremely unlikely \citep{Huang2005HowMammals}. 

Ultimately, this result appears to be confounded by the length of the introns. The X-specific introns have a mean length of $\sim28,000$, an order of magnitude larger than the PAR located introns, which have a mean length of $\sim2,300$. The power of the EOP test is affected by the number of substitution events that distinguish the sequences in an alignment. A longer alignment is more likely to have more substitution events and thus increase the power of the test. Therefore, it seems most likely that this result simply reflects the relative power of each test, in turn, the length of the introns. 

\section{Is the Human genome at equilibrium?}

I now turn my attention to the evolution of the human genome. Unlike the application to empirical controls, there was no expectation of the level or form of mutation disequilibrium. 

\subsubsection{There is substantial evidence for the existence of disequilibrium in the human genome}

I estimated the proportion of human Chromosome 1 which was in disequilibrium to be $\sim50\%$ for intronic sequence and $\sim68\%$ for CDS (Section \ref{Human:TOE}). The level of disequilibrium in humans has been previously estimated by \cite{Squartini2008QuantifyingProcess} and again their estimate is significantly higher than mine, at $>99\%$. As explained earlier, their test statistic excessively rejects the null hypothesis as it does not satisfy the asymptotic distribution (Figure \ref{fig:synthetic/chi2/all-seeds}), ultimately causing their estimate to be incorrect. 

It is important to acknowledge that the estimate of the proportion of human CDS in disequilibrium may be imprecise due to underlying differences between each gene. An assumption of the approach of \cite{Storey2003StatisticalStudies} is that each data point is a result of data drawn from the same underlying distribution, in which case, you can expect that the $p$-values are uniformly distributed. When you are not drawing randomly from the same distribution, even when the null is true, that expectation is not met. This is the case for the CDS analysis, where the data generated under the null hypothesis does not fall on the diagonal line (Figure \ref{Human:TOE}), illustrating the impact of differences such as the length of alignment or rate of evolution. The properties of the estimate are not entirely understood as a consequence, however, the departure from theory is small enough that there is still something to be gained from this data. There is this way to determine what the correct proportion is. This would be to randomly draw from the true negative and true positives in different proportions and solve for the observed fraction. 

Acknowledging Simpson's paradox, my results strongly suggest the existence of disequilibrium in more than half of the examined sequences. What these results do not guarantee is that that the disequilibrium is a consequence entirely of changes to mutagenesis. In fact, for the CDS result, I expect that the level of disequilibrium is influenced by selection. However, I can potentially rule it out as a major factor in the intronic sequences. The determination that selection is operating is often based on variability and rates of evolution (ref). The underlying premise of that work is often that the neutral rate is uniform (ref). The challenge to that is the strong evidence that mutagenesis is not uniform (ref). As a consequence, those conclusions are in doubt, until we have a clearer understanding of what the true baseline of neutral processes looks like. What this means is not that everything is neutral, just that until demonstrated otherwise, we cannot establish whether it is neutral or adaptive. 

It seems more likely that most genomic sequence, even within introns or between genes, is not functional. The demonstrated abundance of functional elements outside of protein coding sequences is low \citep{Graur2013OnENCODE} and the average length of protein binding motifs is approximately 10bp \citep{Stewart2012WhyLong}. 10bp is a minor proportion of the typical length of an intron, $\sim7,000$bp \citep{10.1093/database/baw153}. These observations are not definitive, but they argue against the abundance of natural selection being the force behind these transitions. 

In light of the previous arguments, and using our current understanding of the function of intronic sequences, it seems most plausible that the level of disequilibrium in the intronic sequences is largely due to changes to mutagenesis.  This is a significant result, as it suggests that mutation disequilibrium, a phenomenon widely assumed not to exist, is actually a pervasive feature of genomes. Interestingly, it exists between closely related species, and without evidence of a radical change that may have caused it. 

This result has considerable implications for research domains that use substitution models. Not only do the most popular models in the field assume that the process is in equilibrium, but they also assume reversibility - that the process is identical to how it would appear if run in reverse. It is worth noting that reversibility implies equilibrium, but the reverse is not true. Assuming stationarity and time-reversibility for data where this is not the case can weaken conclusions and even introduce bias to results \citep{Kaehler2015}. The capacity to understand the proportion of a data set where the assumption of equilibrium is untrue, allowing for a more informed choice of substitution model, is a significant implication of my methods.  

\subsubsection{The magnitude of disequilibrium is heterogeneous along a chromosome }

The $\delta_\nabla$ statistic provides an understanding of the spatial organisation of disequilibrium. The distribution of $\delta_\nabla$ along human Chromosome 1 shows that the magnitude of disequilibrium is heterogeneous (Figure \ref{fig:primate:dconv-manhattan}). There are clear regions where the average magnitude of disequilibrium is higher than others, illustrated best by looking at the first 50M section of introns vs the next 50M section. The distribution of $\delta_\nabla$ in the autosomes of \textit{D. melanogaster} does not have such contrasting sections (Figure \ref{fig:drosophila_d-conv_manhattan}). A possible explanation derives from the high level of $^5$mC in the human genome, even compared to \textit{D. simulans}. 

The spatial distribution of $\delta_\nabla$ has a striking resemblance to the frequency of CpG dinucleotides and the GC content of the chromosome (Figure \ref{fig:primate:dconv-manhattan-appendix}). This is an interesting association that I do not have the data points to resolve. However, a possible explanation builds upon the observation that GC content and the deamination of $^5$mC interact and affect each other \citep{Fryxell2000CytosineIsochores, Mugal2015EvolutionaryGenomes}. The proposed model is that although the spontaneous deamination of $^5$mC reduces the GC content and depletes CpG dinucleotides,  a high GC content actually increases the stability of double stranded DNA, acting as a rate limiting factor for deamination \citep{Mugal2015EvolutionaryGenomes}. This implies that the rate of deamination of $^5$mC should be lower in GC rich regions than in GC poor regions \citep{Mugal2015EvolutionaryGenomes}. This potentially creates a positive feedback look in which low GC regions have a higher rate of deamination, depleting CpG dinucleotides and approaching equilibrium quickly, and the stability of the high GC regions slows both the rate of deamination and thus the convergence to equilibrium. An interesting next step would be to see if this association holds for other human chromosomes. If yes, is it limited to humans, or is it a feature of all species with a high level of methylation? Nevertheless, this association further motivates the implementation of my methods to a dinucleotide process. 

\section{The role of purifying selection in influencing mutation disequilibrium}

Although my experimental design did not explicitly consider selection, there are two results in which its influence is visible. These are in comparisons of introns and CDS of the same gene, and comparisons between autosomes and the X chromosome. To resolve whether disequilibrium is influenced by selection as opposed to being from purely mutagenic origins requires a comparison point. This is provided in the Great Ape data by the paired data structure including introns and CDS of the same gene. I expect the introns and exons of a gene to be subject to essentially the same environment. Except, I expect selection to be operating predominantly on the CDS of a gene, making the introns a selectively neutral benchmark \citep{Graur2013OnENCODE}. 

Where the magnitude of disequilibrium is higher in the CDS than the intronic sequence may be evidence that a gene is under purifying selection. If selection is not operating on the CDS of a gene, I expect the magnitude of disequilibrium to be the same between the sequence types. In the case that strong purifying selection is operating on sequence, then you expect a slower rate of evolution, given existing disequilibrium, this will maintain disequilibrium at a higher magnitude. Therefore, if purifying selection is operating on a gene, I expect the magnitude of disequilibrium to be higher in the CDS than in the intronic sequence.  

In the analysis of the human genome, a proportion of genes appeared to be influenced by negative selection (Figure \ref{fig:primate:dconv-diff}). In these cases, owing to a lesser selective constraint, intronic sequences appeared already closer to equilibrium. The proportion of genes for which this was the case was only a modest majority (62\%). Considering the short divergence time between the species, this is still an encouraging result, indicating that the $\delta_\nabla$ statistic is sensitive to such small differences. Analysis of alignments of humans with more diverged primates would allow a clearer understanding of the relationship between negative selection and the magnitude of disequilibrium within a gene. 

The influence of negative selection was also visible on a chromosomal level in the \textit{Drosophila} data. For hetero-gametic sexes, you expect the chromosome that is hemizygous to be subjected to more stringent natural selection, because recessive deleterious alleles are exposed in the hemizygous sex. Analysis of patterns of evolution in \textit{Drosophila} indeed suggests that the efficiency of negative selection is greater on the X chromosome than the autosomes \citep{Singh2008ContrastingDrosophila}. In both \textit{D. melanogaster} and \textit{D. simulans} the mean magnitude of disequilibrium was significantly higher in the X Chromosome than the autosomes, consistent with a higher level of purifying selection acting on X-linked genes. 

What about the ability to tell whether positive selection is operating on a gene? If there is really strong positive Darwinian selection, then the rate of evolution is faster than the neutral rate. So paradoxically, a  strong selection for change will lose disequilibrium because it will evolve to its equilibrium quicker. This is only if you imagine that the full state space of possible change is obtainable. Exactly how positive selection would be reflected in the methods presented in this thesis is unknown. However, to begin to resolve this would require application to genes for which there is strong evidence that they are under the influence of positive selection. 

\section{The Evolution of mutation}

The existence of mutation disequilibrium must be interpreted alongside the theory of evolution of mutation, as it unifies ideas of mutation, selection and drift \citep{Lynch2008TheEvolution, Lynch2010EvolutionRate}. The theory, proposed by \cite{Lynch2008TheEvolution}, states that mutation rates reflect the balance of selection operating to lower the mutation rate, opposed by random genetic drift \citep{Lynch2010EvolutionRate}. Mutators are selected against through the deleterious changes they create, and antimutators are selected for by the reduction in the deleterious mutation rate they provide. In both cases, selection must overcome the power of random genetic drift. Thus, there exists an equilibrium in each lineage that reflects this balance. In the case that an antimutator poses a sufficient selective advantage, and thus does evolve, this would shift the equilibrium of the mutation rate. 

The loss methylation in \textit{D. melanogaster} is an example of an antimutator allele that has overcome a substantial barrier to lower the mutation rate. Critically, this recent evolution of mutation, and its shift of the equilibrium of the mutation rate, is reflected in the methods I have developed. Assuming the disequilibrium in \textit{D. melanogaster} truly echos the recent evolution of mutation, this application demonstrates that my methods can detect the perturbation of mutation rate equilibrium by the evolution of mutation. This highlights the ability for application across the tree of life to help characterise how often mutators evolve. 

\section{Future Directions}

Altogether, the results of my applications demonstrates the value of my methods in understanding the disequilibrium of DNA sequence evolution, yet the real promise lies in their future application.

The capacity of my methods to reflect a recent antimutator evolution highlights the possibility to corroborate the underlying premises of the theory of mutation. The arguments made by \cite{Lynch2008TheEvolution, Lynch2010EvolutionRate} are supported by two key observations: the inverse relationship between the mutation rate and genome size in microbes, and the negative scaling between mutation rate and effective population size in eukaryotes \citep{Lynch2008TheEvolution, Lynch2010EvolutionRate}. Both these observations are consistent with the random genetic drift opposing the lower bound of mutation rates, because the power of drift is ultimately dependent on the effective genetic population size \citep{Lynch2008TheEvolution, Lynch2010EvolutionRate}. 

Applying $\delta_\nabla$ to a broad range of microbes, I would expect the magnitude of mutation disequilibrium to have an inverse relationship with genome size. A similar prediction could be made about eukaryotes, except mutation disequilibrium is expected to increase with genome size. These hypotheses are based on selection having a greater capacity to select against mutators that would otherwise perturb the equilibrium. This general application will further our understanding of the interplay between between selection, mutation and drift. 
  
The ability to detect an increase in disequilibrium in a sequence translocated to a new environment exposes many avenues of future applications. One possible example is the detection of horizontal gene transfer (HGT) in microbes. Genes obtained by HGT often confer functional advantages (ref). 

A ambitious yet pertinent application would be analysis of the SARS-CoV-2 genome. 

\section{Conclusions}

I this thesis I have presented a set of tools that address the statistical problems of testing for disequilibrium, measuring its magnitude, and whether it is equivalent between samples. Through considered experimental design I demonstrated that the tools are fit for purpose. These methods represent a significant contribution to biology, as they provide the capacity to further understand mutation disequilibrium, and in turn, the complex and dynamic relationship between selection, mutation and drift. 

Using the new statistical tools I have unveiled the status of mutation disequilibrium in three key applications, showing that mutator evolution and translocation are two of possible many of its putative origins. Indeed, even when considering a short time frame, more than half the human genome exhibits evidence of mutation disequilibrium. The capacity of the tools to effectively detect, measure, and distinguish mutation disequilibrium is a truly exciting prospect for future applications. 

