\chapter{Discussion}

\section{Is the Human Genome at Equilibrium?}



The Quantile-Quantile plots suggest that there is considerable evidence for non-equilibriumness. What these results do not guarantee is that that non-equilibriumness is a consequence entirely of changes to mutagenesis. That is a presumption. There is confidence that there is substantial non-equlibriumness on the basis that the plots. acknowledging that we can never be certain because of Simpson's paradox, because what we are analysing is just a single dimension-ed thing. However, what I cannot say with certainty, is the origins. That the reason that there is departure is entirely due to changes in mutagenesis. I will have to make logical statements. 

logical statement: 

Demonstrated abundance of functional elements in the genome outside of protein coding sequence -> Evidence of rate of evolution in intronic sequences is greater than the rate of evolution in exonic sequences from the primate analysis based on the length distribution (there's overlap but its mostly to the right of zero -> as you go out to more divergent primate the introns become less recognisably related to each other, indicating that the rate of evolution of introns is very strong -> so if we use conservation, in terms of preserving the permutation of nucleotides because it encodes function as an indicator of the operation of natural selection -> then it is greatly diminished within a relatively short time period (20 - milltion years) -> these are not definitive, but the argue against the abundance of natural selection being the force behind these transitions.

Impact of Natural Selection \cite{Graur2013OnENCODE}. Essentially, the questions remains open. 
low influence of natural selection in introns. 
The determination that selection is operating is often based on variability and rates of evolution. The underlying premise of that work, is often, that the neutral rate is uniform. The challenge to that is the strong evidence that mutagenesis is not uniform. As a consequence, those conclusions are in doubt, until we have a clearer understanding of what the true baseline of neutral processes look like. What this means is not that everything is natural, just that until demonstrated otherwise, we cannot establish whether it is neutral or adaptive. 



Relying on correlations and inferences about historic processes is fraught, because we cant go back. You need evidence for functional elements, and the best evidence for that is direct genetic manipulation. This is the most compelling evidence.

The level of certainty in a statement I can make is:
It seems more likely that most genomic sequence, even within introns and/or between genes, is not functional. The size of motifs that proteins is quite modest (8-9 nts). 

My experimental design only considers mutagenic processes. Fxy and Drosophila are only about mutation. What we don't have is an explicit consideration of the influence of selection. But I have it indirectly in the form of the third codon position. I could look at 1st vs 3rd codon position to indirectly get at the selective influence. 

Can I rule out changes in selection as a factor in non-equilibriumness. We dont know how natural selection affects the magnitude of non-stationarity. There is almost certainly an inpact of selection, certainly in the protein coding sequence. But, considering the scale of functional elements relative to the length of the data, could pretty confidently rule it out as a major factor in introns. 


There is a proportion of the genome which is at equilibrium, according to the intronic results. There is a different proportion according to the exonic data. This tells you that the genome is heterogeneous in the distribution of non-equilibriumness. This raises a question, is this because of simply different rates of evolution towards the equilibrium across the entire genome, so this what ever has generated this disequilibrium? Or are these localised events? What are we seeing the results of. But I have some methods that can help to start tackle this. 


The assumption that the null hypothesis is uniformly distributed is a very important for the methods of estimating the amount of data that is consistent with the null to be appropriate. In the exonic data, that the null data does not fall on the diagonal line illustrates that there is a striking impact of the data not coming from the same distribution. The estimation procedures have that flax (require uniform distribution). The properties of the estimate are not entirely understood as a consequence of that, and how accurate that is going to be. The departure from theory is small enough that there is still something to be gained from this data, but it means that the estimate is imprecise (we do not know the impact on the procedure for estimating the proportion). 

What you could do, is randomly draw from the true negative and true positives in different proportions and ask how well that fits with your observed data, and try to solve for the fraction. 

There is the prediction that given the intronic data is closer to equilibrium, it will have  lower Convergence estimates. It is, however, difficult to compare between data sets where the properties of the data are different. There is a large impact of sampling error on the statistic. 



\section{Drosophila}

\section{Future Directions}
An important avenue for future work is to figure out a transformation of these statistics that facilitates comparisons between alignments of different lengths. 


\section{Conclusion}