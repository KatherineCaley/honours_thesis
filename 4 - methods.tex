\chapter{Methods}

\section{Markov Models}

\subsection{Expected Number of Substitutions}

\section{Simulation Study}

\subsection{Choosing Seed Alignments}
I require alignments simulated in accordance with a GNS process. One way to obtain data evolving under a GNS process is to exploit the fact that even a non-stationary process will converge to its stationary nucleotide distribution (herein $\pi_{\infty}$) as time goes to infinity. Essentially, I can obtain a GN model from real data, derive its corresponding $\pi_{\infty}$ and use those two things to define a stationary process. The process can be modified to model the background edges as discrete, and I can simulate alignments according to such a specification. This is my chosen simulation method as the resulting alignments are generated from parameters of real data. 

I must consider how the properties of the alignment used for simulation may affect the application of my methods. To address this requires deciding the features of an alignment that may be significant and finding measures that characterise them. I can then choose a selection of alignments that vary systematically by these measures. From each alignment, I can create a data set, resulting in a collection of data sets that differ in a way that reflects the natural variation of real sequence data. Using carefully diversified data allows for a thorough interrogation of my methods and determines how their properties may change when applied to different types of data.

An important measure I require is that of non-stationarity. A direct consequence of a non-stationary process is a change in base composition over time. As follows, an indication of the degree of non-stationarity can be obtained from the difference in compositions between sequences in the same alignment. It is worth noting that studies of compositional data often require representation using Aitchison geometry. This representation allows for a consideration of the vulnerability of a composition to the sample it comes from. However, the composition of bases can also be considered as a probability distribution (i.e., $\pi = [\pi_T,\, \pi_G, \, \pi_C, \, \pi_A]$, such that $\pi_i$ where $i= \mathrm{T}, \mathrm{G}, \mathrm{C}, \mathrm{A}$ is the probability of observing the state $i$). In the case of comparing probability distributions, several measures can be used. My selected measure is an information theoretic measure known as Jensen-Shannon Divergence (JSD). JSD measures the similarity between probability distributions. JSD was chosen as it can accommodate multiple distributions, allowing us to apply it to more than two sequences if needed. Additionally, it has an associated true metric, satisfying important mathematical properties (e.g., the triangle inequality). 

A probability distribution also has an associated information content, measured using entropy. Entropy is a fundamental quantity that indicates, in our case, the evenness of base composition. For example, $\pi =[1,\, 0,\, 0,\, 0]$  (a single nucleotide) has zero entropy whilst  $\pi =[0.25,\, 0.25,\, 0.25,\, 0.25]$ (equifrequent nucleotides) has the maximum possible entropy for 4 states. As a measure of compositional diversity, it captures an essential feature of the mode of evolution. For example, an edge that is highly \gls{strand-asymmetric} would have lower entropy than a \gls{strand-symmetric} edge. The base composition may affect the properties of my developed methods. One example being $T_{50}$ (see \hyperref[aim3]{Aim 3}), for which base composition is used in its computation. Accordingly, the impact of compositions with different entropy is an important feature to consider in the process of method development.

I will approximate the stability of the process using the condition number of the foreground rate matrix. It is important to derive data from numerically stable processes. If a process falls within the scope of being numerically unstable, (meaning computers are poorly equipped to evaluate it using standard settings), I need to be aware of this so I can select a more numerically stable method. An indication of numerical stability is the eigenvector matrix condition number. A matrix condition number is an approximation of the worst case relative change in derivations, for a relative change in the input. For a continuous-time process, the transition rate matrix $\mathrm{Q}$, specifies the instantaneous rates of exchanges between states. $\mathrm{Q}$ is foundational to a given substitution model. The eigendecomposition of $\mathrm{Q}$ is a representation of $\mathrm{Q}$ in terms of its eigenvalues and eigenvectors. Eigendecomposition is fundamental to many methods that I am using and developing (e.g., deriving $\pi_{\infty}$). As an indication of the suitability of a matrix to decomposition, I will use the eigenvector matrix condition number to approximate the numerical stability of a process.

I chose four microbial alignments to generate synthetic data sets and refer to these as seed alignments. As previously stated, I expect that JSD, entropy, and condition number may affect the numerical and or statistical properties of the developed methods. To derive data from numerically stable processes, the seed alignments are chosen from a subset of alignments where the eigenvector condition number is low ($<1.5$). To capture the extent of naturally occurring diversity (measured by JSD) and compositional diversity (measured by entropy), the seed alignments chosen represent the permutations of those extremes. These are: high JSD, low entropy; high JSD, high entropy; low JSD, low entropy; low JSD, high entropy. 

\subsection{Generating Synthetic Alignments that are Stationary, but not Reversible}
I expect the properties of my test to be affected by the number of substitution events that distinguish the sequences in an alignment. The mathematical proof that LRT statistics are $\chi^{2}$ distributed assumes infinite data, that is the alignment is infinitely long. Although it has been shown that a few hundred base pairs can be sufficient for the $\chi^{2}$ to be accurate, this length is ultimately dependent on the number of substitution events \cite{Ota2000AppropriateParameters}. I can increase the number of events in a simulated alignment in two ways, (1) increase the length of the alignment, (2) increase the \gls{branch length} of the tree (increased divergence time). 

It is necessary to determine how my methods are affected by the length of the alignment. I have done this by simulating multiple data sets differing in alignment length for all four seeds. The lengths were chosen to be representative of alignment lengths of my biological application. The shortest sequences I will be using are protein coding genes, for which the average length (using just the 3rd codon position) is about 300bp. For each of the four seeds, I generated data sets of alignment length 300, 3,000, and 30,000.   

It is also necessary to determine the influence of increased branch length. When simulating alignments, I can alter the branch lengths of the generating function. This is a capability that allows me to create synthetic alignments specified by the same process (same $\mathrm{Q}$), but with an increased term of divergence. To address how branch length affects the test, I simulated data sets with an increased branch length for all four seeds. The genetic distance ($d$) (or branch length) is a measure of the expected number of nucleotide substitutions per site. The maximum branch length in my application data sets will be about $d=0.6$. For the long branch simulations, I scaled the tree by a factor of 3. However, branch lengths were capped at $d=0.6$ and the remaining branch lengths were reduced proportionally. 

Each simulation set consisted of 1,000 synthetic alignments. The synthetic alignments are generated from the same function, derived from the parameter estimates from one seed alignment. The simulation process that was performed is described in Figure \ref{fig:seq2sim}. 



\section{Bootstrapping Data}

\section{Data Sets}

\section{Algorithmic Implementation}

