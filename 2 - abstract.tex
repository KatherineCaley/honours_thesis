\thispagestyle{plain}
\begin{center}
    
    \textbf{Abstract}
    
    Most models of sequence divergence assume the composition of nucleotides does not change through time. This assumption requires a state of mutation equilibrium which is almost impossible if the processes affecting mutagenesis change through time. Considerable empirical evidence strongly suggests that this may be incorrect. This honours thesis addresses this possibility through developing the following statistical measures: a test for the existence of mutation disequilibrium, a test of its equivalence and a measurement of the magnitude of mutation disequilibrium. I used careful construction of edge cases with simulated data to establish the consistency of the statistics with theoretical expectations. I applied the statistics to empirical data from cases with striking prior evidence for recent perturbations affecting: an entire genome (loss of DNA methylation in \textit{Drosophila melanogaster}); or, a small genomic segment (\textit{Fxy} in \textit{Mus musculus}). Using paired experimental designs, I show the predicted vast excess of small probabilities from the statistical tests. I further show the statistical measure of magnitude is also elevated in these cases. Applying the methods to Human evolution, I conservatively estimate $>$ 50\% of our genome is in mutation disequilibrium. 


\end{center}