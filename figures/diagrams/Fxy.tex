\begin{figure}[htbp]
\centering
\includegraphics[width=\textwidth]{figures/diagrams/Fxy.pdf}
\caption[Evolutionary History of \textit{Fxy} in Rodents]{\textbf{Evolutionary History of \textit{Fxy} in Rodents}. \textbf{(a)} The \textit{Fxy} gene in \textit{M. musculus} was translocated from a X-specific position to a new position in which it overlaps with the Pseudo-Autosomal Region (PAR). The overlap in \textit{M. musculus} is shown as the shaded region of the gene.  The divergence timescale is indicated in millions of years. In rodents and other mammals \textit{Fxy} is X-specific, suggesting translocation to the PAR in \textit{M. musculus} after divergence with \textit{M. spretus}, between 2-3 million years ago \citep[adapted from Figure 1][]{Galtier2007AdaptationEvolution}. \textbf{(b)} The structure of the \textit{Fxy} gene in \textit{M. musculus} in which the boundary to the PAR falls in intron 3 \citep{Palmer1997AMice}. The $5'$ end of the gene, exons 1-3, are X-specific. The $3'$ end of the gene, exons 4-10, are positioned in the PAR. Note that for simplicity, exons beyond exon 7 are not shown in the structure. }
\label{fig:Fxy}
\end{figure}