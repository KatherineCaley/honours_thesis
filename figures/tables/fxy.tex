\begin{table}[htbp]

\begin{center}
\setstretch{1.6}
\begin{tabularx}{\textwidth}[t]{ 
  | >{\centering\arraybackslash}c 
  | >{\centering\arraybackslash}X 
  | >{\centering\arraybackslash}X  
  | >{\centering\arraybackslash}X  
  | >{\centering\arraybackslash}X  
  | >{\centering\arraybackslash}X  
  | >{\centering\arraybackslash}X | 
  }
\hline
\textbf{{Intron Rank}} & \textbf{{1}} & \textbf{{2}} & \textbf{{3}} & \textbf{{4}} & \textbf{{5}} & \textbf{{6}} \\
\hline
TOE $\hat p$-value & $<0.01$ & $<0.01$ & 0.01 & $<0.01$ & $<0.01$ & $<0.01$ \\
\hline
$\hat\delta_\nabla$ & 0.0019 & 0.0024 & 0.0115 & 0.0490 & 0.0470 & 0.0671 \\
\hline
\end{tabularx}
\end{center}
\caption[The magnitude of mutation disequilibrium is highest in the region of the \textit{Fxy} gene that resides in the PAR]{\textbf{The magnitude of mutation disequilibrium is highest in the region of the \textit{Fxy} gene that resides in the PAR.} The table shows the TOE $\hat p$-value and $\hat\delta_\nabla$ for the first six $5'$ introns of \textit{M. musculus}. Data comes from alignments of \textit{M. musculus}, \textit{M. spretus}, and \textit{R. norvegicus} where \textit{M. musculus} is treated as the foreground edge in model fitting.}
\label{table:nablaFxy}

\end{table}